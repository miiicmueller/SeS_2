% Chapter Template

\chapter{Introduction} % Main chapter title

\label{Chapitre 1} % Change X to a consecutive number; for referencing this chapter elsewhere, use \ref{ChapterX}

\lhead{ \emph{Introduction}} % Change X to a consecutive number; this is for the header on each page - perhaps a shortened title

La sécurité dans les systèmes embarqués, qui sont présents de plus en plus souvent dans un nombre de domaine grandissant, est primordiale pour éviter les abus de personnes malintentionnées (voir éviter les erreurs humaines possibles...). \\

Le but de ces laboratoires était de rendre le système embarqué moins vulnérable, notamment en ce qui concerne les systèmes de fichiers, les mots de passe ainsi que le démarrage de Linux. \\ 

Dans ce laboratoire, nous avons pu appliquer certaines méthodes pour mieux gérer les fichiers et systèmes de fichiers, comme par exemple les systèmes de fichiers en lecture seule, les droits des fichiers et le chiffrement d'un système de fichiers. Nous avons aussi vu comment les mots de passe étaient stockés et comment vérifier que leur stockage est sécurisé. Et enfin nous avons vu comment améliorer la sécurité du système au démarrage avec les droits des démons et des scripts de démarrage.\\


























