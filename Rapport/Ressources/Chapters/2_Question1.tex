% Chapter Template

\chapter{Question 1} % Main chapter title

\label{Question 1} % Change X to a consecutive number; for referencing this chapter elsewhere, use \ref{ChapterX}

\lhead{ \emph{EXT 4}} % Change X to a consecutive number; this is for the header on each page - perhaps a shortened title

%----------------------------------------------------------------------------------------
%	SECTION 1
%----------------------------------------------------------------------------------------
\section{What is the real name for the node file "/dev/root"}

Si on lance la commande suivante :
\begin{lstlisting}[frame=single,style=Console]  % Start your code-block

# ls -al /dev/root
lrwxrwxrwx    1 root     root             9 Jan  1 00:00 /dev/root -> mmcblk0p2
\end{lstlisting}

On remarque que "root" est un lien symbolique sur /dev/mmcblk0p2. 

\section{What are the major and minor number for the "/dev/root" node file}

Par analogie, on regarde les informations sur le fichier pointé :

\begin{lstlisting}[frame=single,style=Console]  % Start your code-block

ls -al /dev/mmcblk0p2 
brw-rw----    1 root     root      179,   2 Jan  1 00:00 /dev/mmcblk0p2
\end{lstlisting}

On voit le numéro majeur est 179 et le numéro mineur 2. 

\section{How the kernel knows that the rootfs in in the second partition}

Le noyau sait que notre "rootfs" est sur "mmcblk0\textbf{p2}", car c'est un paramètre de boot que nous avons configuré dans "u-boot".
