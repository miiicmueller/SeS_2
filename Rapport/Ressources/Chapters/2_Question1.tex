% Chapter Template

\chapter{Question 1} % Main chapter title

\label{Question 1} % Change X to a consecutive number; for referencing this chapter elsewhere, use \ref{ChapterX}

\lhead{ \emph{EXT 4}} % Change X to a consecutive number; this is for the header on each page - perhaps a shortened title

%----------------------------------------------------------------------------------------
%	SECTION 1
%----------------------------------------------------------------------------------------
\section{What is the real name for the node file "/dev/root"}

Sur la cible, la commande \textbf{mount} fourni les résultats suivants :
\begin{lstlisting}[label=lst:mount odroid]
# mount
rootfs on / type rootfs (rw)
/dev/root on / type ext4 (rw,relatime,data=ordered)
devtmpfs on /dev type devtmpfs (rw,relatime,size=765136k,nr_inodes=120937,mode=755)
proc on /proc type proc (rw,relatime)
devpts on /dev/pts type devpts (rw,relatime,gid=5,mode=620)
tmpfs on /dev/shm type tmpfs (rw,relatime,mode=777)
tmpfs on /tmp type tmpfs (rw,relatime)
sysfs on /sys type sysfs (rw,relatime)
\end{lstlisting}
De ces quelque lignes, il vient que /dev/root pointe vers la racine du système, à savoir /. Ceci provient des paramètres passés au noyau (\ref{lst:parameters}) lors de son démarrage et correspond physiquement à la partition mmcblk0p2 de la carte \usd. Ce point de montage (/) possède un autre nom tel que rootfs (première ligne de listing \ref{lst:mount odroid}). En effet, ce nom provient du formatage de la carte \usd réalisé lors d'un précédent laboratoire lequel nomme cette partition \textbf{rootfs}.

\begin{lstlisting}[caption=Paramètres du noyau, label=lst:parameters]
Kernel command line: console=ttySAC2,115200n8 earlyprintk debug root=/dev/mmcblk0p2 rw rootwait rootfstype=ext4 ip=192.168.0.11:192.168.0.4:192.168.0.4:255.255.255.0:odroidxu3:eth0:off
\end{lstlisting}

Pour vérifier cette théorie, l'utilisation de la commande readlink donne :
\begin{lstlisting}
# readlink -f /dev/root
/dev/mmcblk0p2
\end{lstlisting}
ou encore:
\begin{lstlisting}
# ls -l /dev/root
lrwxrwxrwx    1 root     root             9 Jan  1 00:01 /dev/root -> mmcblk0p2
\end{lstlisting}

\section{What are the major and minor number for the "/dev/root" node file}
Cette dernière commande affiche une information de plus à savoir les numéro majeur et mineur du fichier en question. Ici, \textbf{/dev/root} ne possède pas de numéro majeur (major number) ni mineur car il s'agit d'un lien. Ces numéros identifient les pilotes sous Linux. De plus, les fichiers spéciaux identifiant un pilote de périphérique de type caractère présente la lettre c devant la ligne correspondante. Ici, \textbf{/dev/root} est un lien symbolique vers \textbf{/dev/mmcblk0p2} qui lui-même est un pilote de type \textbf{block}. Ceci signifie que le noyau écrit les données sur celui-ci par multiple de la taille du bloc. \textbf{/dev/mmcblk0p2} possède le numéro majeur 179 et le numéro mineur 2.
\begin{lstlisting}[style=Bash]
# ls -l /dev/root
lrwxrwxrwx    1 root     root             9 Jan  1 00:01 /dev/root -> mmcblk0p2
# ls -l /dev/mmcblk0p2
brw-rw----    1 root     root      179,   2 Jan  1 00:00 /dev/mmcblk0p2
\end{lstlisting}

Ainsi, le noyau connaît "l'adresse" de \textbf{rootfs} grâce au paramètre \textbf{root=/dev/mmcblk0p2} qui lui est passé au démarrage.
