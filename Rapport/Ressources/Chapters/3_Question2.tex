% Chapter Template

\chapter{Question 2} % Main chapter title

\label{Question 2} % Change X to a consecutive number; for referencing this chapter elsewhere, use \ref{ChapterX}

\lhead{ \emph{EXT 4}} % Change X to a consecutive number; this is for the header on each page - perhaps a shortened title

%----------------------------------------------------------------------------------------
%	SECTION 1
%----------------------------------------------------------------------------------------
\section{Montage de usrfs}
La commande suivante permet de monter le système de fichier \textbf{usrfs} au point de montage \textbf{/mnt/usr}.
\begin{lstlisting}[style=Console]
# mkdir /mnt/usr
# mount /dev/mmcblk0p3 /mnt/usr -t ext4
# mount
	...
	/dev/mmcblk0p3 on /mnt/usr type ext4 (rw,relatime,data=ordered)
\end{lstlisting}
Puis vient le fichier fstab qui sert à monter cette partition au démarrage du noyau de façon automatique. La ligne suivante doit être ajoutée au fichier afin de permettre ce montage automatique.
\begin{lstlisting}[frame=single,style=Console]  % Start your code-block

# cat /etc/fstab 
# /etc/fstab: static file system information.
#
# <file system> <mount pt>     <type>	<options>         <dump> <pass>
/dev/root       /              ext2	rw,noauto         0      1
/dev/mmcblk0p3	/mnt/usr       ext4     defaults	  0      1
proc		/proc	       proc     defaults	  0	 0
devpts		/dev/pts       devpts   defaults,gid=5,mode=620	  0	 0
tmpfs           /dev/shm       tmpfs    mode=0777         0      0
tmpfs           /tmp           tmpfs    mode=1777         0      0
sysfs		/sys	       sysfs    defaults	  0	 0
\end{lstlisting}
Pour le contenu du fichier \textbf{fstab}, se référer à l'annexe \ref{an:fstab}
