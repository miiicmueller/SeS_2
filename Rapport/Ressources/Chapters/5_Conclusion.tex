% Chapter Template

\chapter{Conclusion} % Main chapter title

\label{Conclusion} % Change X to a consecutive number; for referencing this chapter elsewhere, use \ref{ChapterX}

\lhead{ \emph{Conclusion}} % Change X to a consecutive number; this is for the header on each page - perhaps a shortened title


Ce laboratoire nous a permis de mettre en pratique la théorie vue en cours ainsi que de rafraîchir nos connaissances précédentes sur Linux embarqué. Les systèmes embarqué étant "limités" (cette affirmation est aujourd'hui à moitié vraie... Notre plate-forme est tout de même dotée d'un octo-coeur ARM et de beaucoup de mémoire vive. Mais ce n'est pas forcément le cas de tous les systèmes sur le marché) en performances nécessitent un traitement particulier quand à leur sécurité.\\ 

On peut facilement prendre ces systèmes et en extraire les données. Une partie de la sécurité peut être crée à la conception hardware du système (protection des ports JTAG, etc...), mais le reste doit être géré au niveau du software. \\

Nous avons vu cette fois-ci les diverse protections du système de fichiers. Pas seulement au niveau de la confidentialité des données en chiffrant la/les partitions, mais aussi au niveau des droits sur celles-ci. Nous avons vu qu'un programme vulnérables qui possède des droits "root" peut permettre à un adversaire de prendre le contrôle sur le système.\\

Le filtrage du réseau aurait fait partie intégrante du laboratoire, mais pour des question de temps (comme la sécurité des filesystems à pris beaucoup de temps, cette partie nous à été dispensée.

