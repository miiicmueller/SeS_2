% Chapter Template

\chapter{Question 3} % Main chapter title

\label{Question 3} % Change X to a consecutive number; for referencing this chapter elsewhere, use \ref{ChapterX}

\lhead{ \emph{EXT 4}} % Change X to a consecutive number; this is for the header on each page - perhaps a shortened title

Sur la cible, la commande mount fourni les résultats suivants :
\begin{lstlisting}[label=lst:mount odroid]
# mount
rootfs on / type rootfs (rw)
/dev/root on / type ext4 (rw,relatime,data=ordered)
devtmpfs on /dev type devtmpfs (rw,relatime,size=765136k,nr_inodes=120937,mode=755)
proc on /proc type proc (rw,relatime)
devpts on /dev/pts type devpts (rw,relatime,gid=5,mode=620)
tmpfs on /dev/shm type tmpfs (rw,relatime,mode=777)
tmpfs on /tmp type tmpfs (rw,relatime)
sysfs on /sys type sysfs (rw,relatime)
\end{lstlisting}
De ces quelque lignes, il vient que /dev/root pointe vers la racine du système, à savoir /. Ceci provient des paramètres passés au noyau (\ref{lst:parameters}) lors de son démarrage et correspond physiquement à la partition mmcblk0p2 de la carte \usd. Ce point de montage (/) possède un autre nom tel que rootfs (première ligne de listing \ref{lst:mount odroid}). En effet, ce nom provient du formatage de la carte \usd réalisé lors d'un précédent laboratoire lequel nomme cette partition \textbf{rootfs}.

\begin{lstlisting}[caption=Paramètres du noyau, label=lst:parameters]
Kernel command line: console=ttySAC2,115200n8 earlyprintk debug root=/dev/mmcblk0p2 rw rootwait rootfstype=ext4 ip=192.168.0.11:192.168.0.4:192.168.0.4:255.255.255.0:odroidxu3:eth0:off
\end{lstlisting}

Pour vérifier cette théorie, l'utilisation de la commande readlink est utile :
\begin{lstlisting}
# readlink -f /dev/root
/dev/mmcblk0p2
\end{lstlisting}
ou encore:
\begin{lstlisting}
# ls -l /dev/root
lrwxrwxrwx    1 root     root             9 Jan  1 00:01 /dev/root -> mmcblk0p2
\end{lstlisting}
Cette dernière commande affiche une information de plus à savoir les numéro majeur et mineur du fichier en question. Ici, \textbf{/dev/root} ne possède pas de numéro majeur (major number) et possède le numéro mineur 9. Ces numéros identifient les pilotes de type caractères sous Linux. De plus, les fichiers spéciaux identifiant un pilote de périphérique de type caractère présente la lettre c devant la ligne correspondante. Ici, \textbf{/dev/root} est un lien symbolique vers \textbf{/dev/mmcblk0p2} qui lui-même est un pilote de type \textbf{block}. Ceci signifie que le noyau écrit les données sur celui-ci par multiple de la taille du bloc. \textbf{/dev/mmcblk0p2} possède le numéro majeur 179 et le numéro mineur 2.
\begin{lstlisting}[style=Bash]
# ls -l /dev/root
lrwxrwxrwx    1 root     root             9 Jan  1 00:01 /dev/root -> mmcblk0p2
# ls -l /dev/mmcblk0p2
brw-rw----    1 root     root      179,   2 Jan  1 00:00 /dev/mmcblk0p2
\end{lstlisting}

Ainsi, le noyau connaît "l'adresse" de \textbf{rootfs} grâce au paramètre \textbf{root=/dev/mmcblk0p2} qui lui est passé au démarrage.

\subsection{Montage de usrfs}
La commande suivante permet de monter le système de fichier \textbf{usrfs} au point de montage \textbf{/mnt/usr}.
\begin{lstlisting}[style=Bash]
# mount /dev/mmcblk0p3 /mnt/usr -t ext4
\end{lstlisting}
Puis vient le fichier fstab qui sert à moonter cette partition au démarrage du noyau de façon automatique. La ligne suivante doit être ajoutée au fichier afin de permettre ce montage automatique.
\begin{lstlisting}
/dev/mmcblk0p3  /mnt/usr        ext4    defaults          0      0
\end{lstlisting}
Pour le contenu du fichier \textbf{fstab}, se référer à l'annexe \ref{an:fstab}

%----------------------------------------------------------------------------------------
%	SECTION 1
%----------------------------------------------------------------------------------------
\section{Performances de EXT4}

La page de manuel de la commande mount donne ceci:
\begin{lstlisting}[style=Bash]
data={journal|ordered|writeback}
              Specifies the journalling mode for file data.  Metadata is always journaled.  To use modes other than ordered on the root filesystem, pass the mode to the kernel  as  boot  parameter,  e.g.
              rootflags=data=journal.

              journal
                     All data is committed into the journal prior to being written into the main filesystem.

              ordered
                     This is the default mode.  All data is forced directly out to the main file system prior to its metadata being committed to the journal.

              writeback
                     Data  ordering  is  not  preserved - data may be written into the main filesystem after its metadata has been committed to the journal.  This is rumoured to be the highest-throughput
                     option.  It guarantees internal filesystem integrity, however it can allow old data to appear in files after a crash and journal recovery.

       barrier=0 / barrier=1
              This enables/disables barriers.  barrier=0 disables it, barrier=1 enables it.  Write barriers enforce proper on-disk ordering of journal commits, making volatile disk write caches  safe  to
              use,  at some performance penalty.  The ext3 filesystem does not enable write barriers by default.  Be sure to enable barriers unless your disks are battery-backed one way or another.  Oth‐
              erwise you risk filesystem corruption in case of power failure.
\end{lstlisting}

Pour la réalisation de cet exercice, il a été préalablement nécessaire d'installer un cross-compilateur pour Odroid sur la machine hôte utilisée. Ainsi, l'installation a été réalisée à l'aide des commandes suivantes:
\begin{lstlisting}[style=Bash]
cd /usr/local/arm
wget https://launchpadlibrarian.net/129963014/gcc-linaro-arm-linux-gnueabihf-4.7-2013.01-20130125_linux.tar.xz
xz -d gcc-linaro-arm-linux-gnueabihf-4.7-2013.01-20130125_linux.tar.xz 
tar -xvf gcc-linaro-arm-linux-gnueabihf-4.7-2013.01-20130125_linux.tar
ln -s gcc-linaro-arm-linux-gnueabihf-4.7-2013.01-20130125_linux /usr/local/arm/toolchain
\end{lstlisting}

Pour cet exercice il a fallu au préalable installer et compiler l'application \textbf{tune2fs} à l'aide de buildroot. Le listing \ref{lst:tune2fs} illustre la sélection des packages utilisés.
\begin{lstlisting}[style=Bash]

\end{lstlisting}

\section{SquashFS}
Le système de fichier Squash est un système de type lecture-seule et compressé. Il supporte des tailles de blocs jusqu'à 1Mo afin d'optimiser la compression des fichiers. Pour pouvoir l'utiliser sous Linux, il est au préalable nécessaire de modifier la compilation du kernel afin de le lui faire supporter. Ainsi, l'exécution et la configuration sont données:

\begin{lstlisting}
make linux-menuconfig
    File systems  --->
[*] Miscellaneous filesystems  --->
<*>   SquashFS 4.0 - Squashed file system support
\end{lstlisting}

Afin de réaliser correctement cet exercice, quatre images squashFS ont été créées à l'aide des fichiers et dossiers générés par buildroot lors de la compilation du Kernel. Le script permettant la création des images est donné en annexe \ref{an:script squashfs}. Ainsi, les images obtenues ainsi que leur taille respectives sont données dans le tableau ci-dessous:
\begin{table}[H]
	\centering
	\begin{tabular}{lrp{10cm}}
	Fichier & Taille [ko] & Description\\
	squashfs.default & 14487.5 & image squashfs sans option de compression\\
	squashfs.noD & 26771.4 & image squashfs sans compression des blocs de données\\
	squashfs.noI & 14553.1 & image squashfs sans compression de la table des i-noeuds\\
	squashfs.noF & 24772.6 & image squashfs sans compression des blocs fragmentés
	\end{tabular}
	\caption{\label{table:squashfs size}Résumé de la taille de chaque image squashfs}
\end{table}

\begin{lstlisting}[style=Bash]
# cd /mnt/usr/
# ls
lost+found        squashfs.noD      squashfs.noI
squashfs.default  squashfs.noF
# mkdir ../sqfs
# mount squashfs.default /mnt/sqfs -t squashfs -o loop
# mount
rootfs on / type rootfs (rw)
/dev/root on / type ext4 (rw,relatime,data=ordered)
devtmpfs on /dev type devtmpfs (rw,relatime,size=765120k,nr_inodes=120929,mode=755)
proc on /proc type proc (rw,relatime)
devpts on /dev/pts type devpts (rw,relatime,gid=5,mode=620)
tmpfs on /dev/shm type tmpfs (rw,relatime,mode=777)
tmpfs on /tmp type tmpfs (rw,relatime)
sysfs on /sys type sysfs (rw,relatime)
/dev/mmcblk0p3 on /mnt/usr type ext4 (rw,relatime,data=ordered)
/dev/loop0 on /mnt/sqfs type squashfs (ro,relatime)
# ls -la /mnt/sqfs
total 1
drwxr-xr-x   18 root     root           256 Apr 30  2015 .
drwxr-xr-x    4 root     root          1024 Jan  1 00:02 ..
drwxrwxr-x    2 root     root           993 Apr 30  2015 bin
drwxr-xr-x    3 root     root            52 Apr 14  2015 dev
drwxr-xr-x    7 root     root           439 Apr 30  2015 etc
drwxrwxr-x    3 root     root            26 Mar  5  2015 home
drwxrwxr-x    5 root     root           953 Apr 30  2015 lib
lrwxrwxrwx    1 root     root             3 Apr 14  2015 lib32 -> lib
lrwxrwxrwx    1 root     root            11 Apr 14  2015 linuxrc -> bin/busybox
drwx------    2 root     root             3 Apr 30  2015 lost+found
drwxrwxr-x    2 root     root             3 Mar  5  2015 media
drwxrwxr-x    2 root     root             3 Feb  3  2015 mnt
drwxrwxr-x    2 root     root             3 Feb  3  2015 opt
drwxrwxr-x    2 root     root             3 Feb  3  2015 proc
drwx------    2 root     root            77 Mar  5  2015 root
drwxrwxr-x    2 root     root           744 Apr 30  2015 sbin
drwxrwxr-x    2 root     root             3 Mar  5  2015 sys
drwxrwxrwt    2 root     root             3 Feb  3  2015 tmp
drwxrwxr-x    7 root     root           102 Apr 30  2015 usr
drwxrwxr-x    6 root     root            63 Apr 14  2015 var
# mkdir /mnt/sqfs/test
mkdir: can't create directory '/mnt/sqfs/test': Read-only file system
\end{lstlisting}

\begin{lstlisting}[style=Bash,caption=squashfs.noD]
# mount squashfs.noD /mnt/sqfs -t squashfs -o loop
# mount
rootfs on / type rootfs (rw)
/dev/root on / type ext4 (rw,relatime,data=ordered)
devtmpfs on /dev type devtmpfs (rw,relatime,size=765120k,nr_inodes=120929,mode=755)
proc on /proc type proc (rw,relatime)
devpts on /dev/pts type devpts (rw,relatime,gid=5,mode=620)
tmpfs on /dev/shm type tmpfs (rw,relatime,mode=777)
tmpfs on /tmp type tmpfs (rw,relatime)
sysfs on /sys type sysfs (rw,relatime)
/dev/mmcblk0p3 on /mnt/usr type ext4 (rw,relatime,data=ordered)
/dev/loop0 on /mnt/sqfs type squashfs (ro,relatime)
# ls -la /mnt/sqfs
total 1
drwxr-xr-x   18 root     root           256 Apr 30  2015 .
drwxr-xr-x    4 root     root          1024 Jan  1 00:02 ..
drwxrwxr-x    2 root     root           993 Apr 30  2015 bin
drwxr-xr-x    3 root     root            52 Apr 14  2015 dev
drwxr-xr-x    7 root     root           439 Apr 30  2015 etc
drwxrwxr-x    3 root     root            26 Mar  5  2015 home
drwxrwxr-x    5 root     root           953 Apr 30  2015 lib
lrwxrwxrwx    1 root     root             3 Apr 14  2015 lib32 -> lib
lrwxrwxrwx    1 root     root            11 Apr 14  2015 linuxrc -> bin/busybox
drwx------    2 root     root             3 Apr 30  2015 lost+found
drwxrwxr-x    2 root     root             3 Mar  5  2015 media
drwxrwxr-x    2 root     root             3 Feb  3  2015 mnt
drwxrwxr-x    2 root     root             3 Feb  3  2015 opt
drwxrwxr-x    2 root     root             3 Feb  3  2015 proc
drwx------    2 root     root            77 Mar  5  2015 root
drwxrwxr-x    2 root     root           744 Apr 30  2015 sbin
drwxrwxr-x    2 root     root             3 Mar  5  2015 sys
drwxrwxrwt    2 root     root             3 Feb  3  2015 tmp
drwxrwxr-x    7 root     root           102 Apr 30  2015 usr
drwxrwxr-x    6 root     root            63 Apr 14  2015 var
# mkdir /mnt/sqfs/test
mkdir: can't create directory '/mnt/sqfs/test': Read-only file system
\end{lstlisting}

\begin{lstlisting}[style=Bash,caption=squashfs.noF]
# mount squashfs.noF /mnt/sqfs -t squashfs -o loop
# mount
rootfs on / type rootfs (rw)
/dev/root on / type ext4 (rw,relatime,data=ordered)
devtmpfs on /dev type devtmpfs (rw,relatime,size=765120k,nr_inodes=120929,mode=755)
proc on /proc type proc (rw,relatime)
devpts on /dev/pts type devpts (rw,relatime,gid=5,mode=620)
tmpfs on /dev/shm type tmpfs (rw,relatime,mode=777)
tmpfs on /tmp type tmpfs (rw,relatime)
sysfs on /sys type sysfs (rw,relatime)
/dev/mmcblk0p3 on /mnt/usr type ext4 (rw,relatime,data=ordered)
/dev/loop0 on /mnt/sqfs type squashfs (ro,relatime)
# ls -la /mnt/sqfs
total 1
drwxr-xr-x   18 root     root           256 Apr 30  2015 .
drwxr-xr-x    4 root     root          1024 Jan  1 00:02 ..
drwxrwxr-x    2 root     root           993 Apr 30  2015 bin
drwxr-xr-x    3 root     root            52 Apr 14  2015 dev
drwxr-xr-x    7 root     root           439 Apr 30  2015 etc
drwxrwxr-x    3 root     root            26 Mar  5  2015 home
drwxrwxr-x    5 root     root           953 Apr 30  2015 lib
lrwxrwxrwx    1 root     root             3 Apr 14  2015 lib32 -> lib
lrwxrwxrwx    1 root     root            11 Apr 14  2015 linuxrc -> bin/busybox
drwx------    2 root     root             3 Apr 30  2015 lost+found
drwxrwxr-x    2 root     root             3 Mar  5  2015 media
drwxrwxr-x    2 root     root             3 Feb  3  2015 mnt
drwxrwxr-x    2 root     root             3 Feb  3  2015 opt
drwxrwxr-x    2 root     root             3 Feb  3  2015 proc
drwx------    2 root     root            77 Mar  5  2015 root
drwxrwxr-x    2 root     root           744 Apr 30  2015 sbin
drwxrwxr-x    2 root     root             3 Mar  5  2015 sys
drwxrwxrwt    2 root     root             3 Feb  3  2015 tmp
drwxrwxr-x    7 root     root           102 Apr 30  2015 usr
drwxrwxr-x    6 root     root            63 Apr 14  2015 var
# mkdir /mnt/sqfs/test
mkdir: can't create directory '/mnt/sqfs/test': Read-only file system
\end{lstlisting}

\begin{lstlisting}[style=Bash,caption=squashfs.noI]
# mount squashfs.noI /mnt/sqfs -t squashfs -o loop
# mount
rootfs on / type rootfs (rw)
/dev/root on / type ext4 (rw,relatime,data=ordered)
devtmpfs on /dev type devtmpfs (rw,relatime,size=765120k,nr_inodes=120929,mode=755)
proc on /proc type proc (rw,relatime)
devpts on /dev/pts type devpts (rw,relatime,gid=5,mode=620)
tmpfs on /dev/shm type tmpfs (rw,relatime,mode=777)
tmpfs on /tmp type tmpfs (rw,relatime)
sysfs on /sys type sysfs (rw,relatime)
/dev/mmcblk0p3 on /mnt/usr type ext4 (rw,relatime,data=ordered)
/dev/loop0 on /mnt/sqfs type squashfs (ro,relatime)
# ls -la /mnt/sqfs
total 1
drwxr-xr-x   18 root     root           256 Apr 30  2015 .
drwxr-xr-x    4 root     root          1024 Jan  1 00:02 ..
drwxrwxr-x    2 root     root           993 Apr 30  2015 bin
drwxr-xr-x    3 root     root            52 Apr 14  2015 dev
drwxr-xr-x    7 root     root           439 Apr 30  2015 etc
drwxrwxr-x    3 root     root            26 Mar  5  2015 home
drwxrwxr-x    5 root     root           953 Apr 30  2015 lib
lrwxrwxrwx    1 root     root             3 Apr 14  2015 lib32 -> lib
lrwxrwxrwx    1 root     root            11 Apr 14  2015 linuxrc -> bin/busybox
drwx------    2 root     root             3 Apr 30  2015 lost+found
drwxrwxr-x    2 root     root             3 Mar  5  2015 media
drwxrwxr-x    2 root     root             3 Feb  3  2015 mnt
drwxrwxr-x    2 root     root             3 Feb  3  2015 opt
drwxrwxr-x    2 root     root             3 Feb  3  2015 proc
drwx------    2 root     root            77 Mar  5  2015 root
drwxrwxr-x    2 root     root           744 Apr 30  2015 sbin
drwxrwxr-x    2 root     root             3 Mar  5  2015 sys
drwxrwxrwt    2 root     root             3 Feb  3  2015 tmp
drwxrwxr-x    7 root     root           102 Apr 30  2015 usr
drwxrwxr-x    6 root     root            63 Apr 14  2015 var
# mkdir /mnt/sqfs/test
mkdir: can't create directory '/mnt/sqfs/test': Read-only file system
\end{lstlisting}

\section{Partition squashfs}
Pour réaliser cet exercice, le script de l'exercice 3, laboratoire \#1 (voir annexe \ref{an:script squashfs}) a été exécuté lorsque la carte \usd se trouvait dans le lecteur. Le système cible à ensuite pu être démarré sachant que les variables d'environnement uboot ont été modifiées à l'aide des commandes suivantes.


Après avoir modifié le contenu du fichier /etc/fstab tel que donné ci-dessous
\begin{lstlisting}
# /etc/fstab: static file system information.
#
# <file system> <mount pt>     <type>   <options>         <dump> <pass>
/dev/root       /              ext2     rw,noauto         0      1
proc            /proc          proc     defaults          0      0
devpts          /dev/pts       devpts   defaults,gid=5,mode=620   0      0
tmpfs           /dev/shm       tmpfs    mode=0777         0      0
tmpfs           /tmp           tmpfs    mode=1777         0      0
sysfs           /sys           sysfs    defaults          0      0
/dev/mmcblk0p3  /mnt/usr       ext4     defaults          0      0
/dev/mmcblk0p4  /mnt/sqfs      squashfs defaults          0      0
\end{lstlisting}

les partitions ont pu être montées après création de leur point de montage sur la partition \textbf{rootfs}.
\begin{lstlisting}
# mkdir /mnt/usr
# mkdir /mnt/sqfs
\end{lstlisting}

Ainsi, le montage du système de fichier \textbf{squashfs} donne les résultats suivants :
\begin{lstlisting}
# mount -a
[  110.733993] [c7] EXT4-fs (mmcblk0p3): mounted filesystem with ordered data mode. Opts: (null)
Jan  1 00:52:27 odroidxu3 user.info kernel: [  110.733993] [c7] EXT4-fs (mmcblk0p3): mounted filesystem with ordered data mode. Opts: (null)
# ls /mnt/usr/
lost+found
# ls /mnt/sqfs/
bin         home        linuxrc     mnt         root        tmp
dev         lib         lost+found  opt         sbin        usr
etc         lib32       media       proc        sys         var
# mount
rootfs on / type rootfs (rw)
/dev/root on / type ext4 (rw,relatime,data=ordered)
devtmpfs on /dev type devtmpfs (rw,relatime,size=765120k,nr_inodes=120929,mode=755)
proc on /proc type proc (rw,relatime)
devpts on /dev/pts type devpts (rw,relatime,gid=5,mode=620)
tmpfs on /dev/shm type tmpfs (rw,relatime,mode=777)
tmpfs on /tmp type tmpfs (rw,relatime)
sysfs on /sys type sysfs (rw,relatime)
/dev/mmcblk0p3 on /mnt/usr type ext4 (rw,relatime,data=ordered)
/dev/mmcblk0p4 on /mnt/sqfs type squashfs (ro,relatime)
\end{lstlisting}

\subsection{Vérification de lecture seule}
Un premier test de lecture seule a été réalisé à l'aide de la commande mkdir qui permet de créer un dossier. Cette commande a échoué sur le système de fichier cible (/mnt/sqfs:squashfs) qui signale "\textbf{Read-only file system}".
\begin{lstlisting}
# mkdir /mnt/sqfs/test
mkdir: can't create directory '/mnt/sqfs/test': Read-only file system
\end{lstlisting}

Un second test a été réalisé à l'aide de l'édition du fichier \textbf{/etc/fstab} se trouvant sur le système cible. Le logiciel \textbf{vi} signale
\begin{lstlisting}[style=Bash]
- etc/fstab [Readonly] 1/9 11%
\end{lstlisting}
lors de l'édition. Le test a ensuite été poursuivi lors de la modification de ce fichier puis de sa sauvegarde. En effet, vi signale
\begin{lstlisting}[style=Bash]
'etc/fstab' is read only
\end{lstlisting}
lors de la commande
\begin{lstlisting}[style=Bash]
:wq
\end{lstlisting}
